\subsection{Justification of Chosen Language}
\paragraph{}
As I wanted a language that I had years of experience with, I had 3 options to choose from - C++, python, and the bash shell. The bash shell, however, is not really suited to writing whole programs with - especially large ones such as the one that I was writing, as bash shell code is rather hard to understand.

\paragraph{}
This left me with C++ and python. In this case, it was clear that python was the obvious choice, as the only advantage C++ has over python is the increase in speed, something that doesn't matter so much on a web server. Python had the clear advantage of being extremely popular for web servers, with many useful libraries for it, some of which I had experience with already.

\paragraph{}
I also had to decide between python2 and python3. I chose python3 for several reasons. Firstly, because it is designed to make some of the functions I often use faster, such as the range and map functions returning iterables rather than lists. Secondly, because the syntax is nicer, such as being able to use input instead of raw\_input, and print being a function rather than a statement. Thirdly, there are some really nice new features such as function annotations, which allow you to declare data types in the return value of functions, which makes it a lot easier to check whether code is right, such as in the code below.

\begin{minted}[frame=single,linenos,mathescape]{python}
def get_keys(a_dict: '{str: int}') -> '[str]':
    """docstring here"""
    return 'string here'
\end{minted}

\paragraph{}
In the example above, in order to test whether, for example, you returned the correct type, you could simply add a decorator to the function while testing. For me, however, the major bonus of showing the return type is that the IDE that I was using (pycharm) can autocomplete for you, and if it knows what the type of a variable returned from a function is, it knows what methods are available to it.