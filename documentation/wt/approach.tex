\subsection{Recommendation of Approach to Solving the Problem}
\paragraph{}
I believe that the best way to handle the problem is to create a website in which teachers can collaborate to create the best resources possible for students, and one in which they can create it once and make only slight modifications when they run the course again for a new set of students (or even just use a course that someone else has made as a template). You should be able to create assignments, hand in assignments, and mark them all on this website. It should also be accessible on mobile devices such as tablets and smartphones, because many teachers have ipads which they use at school.

\todo{do they mean method to solve problem (website) or development approach}
\paragraph{}
As this is an extremely complex program, despite there only being 1 person on the development team, I am going to use the structured approach to create this program. 

\paragraph{}
Firstly, I will design the interaction between each part of the system (a "part" might be, for example, a function for a model). Within each part, I will have an extremely large set of functions. For this reason, before I start coding, I will first plan out how to test it, and draw up a system flowchart or write stubs for the part, describing the inputs and outputs of each function. Next, I will create test data. For example, to test the login, I would write several tests which tested whether the password was checked correctly, whether invalid usernames such as being blank were allowed, etc.

\paragraph{}
Before I start coding any function, I will ensure I know the expected inputs and outputs. If it is a complex function, I may come up with a proof of correctness and some pseudocode for it in order to eliminate bugs before I start coding. Finally, I will start coding the problem