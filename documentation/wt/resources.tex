\subsection{Outline of Resources Available for your Reference and Assistance}
There are a few libraries which I will be using to make my job easier, both for coding and for deployment.

\subsubsection{Google appengine}
Appengine is one of the many web frameworks available for python. It is relatively easy to learn, has an ORM (stores database tables as classes, which makes it a lot nicer), and is extremely scaleable due to an asynchronous web server and database. It is also extremely easy to deploy on google's servers (who have a limited free service which you could upgrade for a cost).
	
\subsubsection{Jinja2}
Jinja2 is a templating language for python, used to write HTML. It is both extremely simple and powerful, and is fast, widely used, and secure, making it very useful for me. It also integrates with google app engine, making it even more useful.

\subsubsection{Bootstrap}
Bootstrap is a css stylesheet which adds classes so you can make your website look nice easily.

\subsubsection{Jquery}
Jquery is a Javascript library to make things like document traversal, document manipulation, and event handling much simpler which works across many different browsers. I also used a plugin for Jquery called "Jquery form plugin", which makes AJAX requests easier.

\subsubsection{Github}
Github is a website which uses git to do version control on your code. On top of that, it has many nice features, such as issue tracking, which I used to ensure I remembered what I needed to do and what was currently broken in my project.

\subsubsection{Codeship}
Codeship (\url{codeship.io}) is a website which automatically deploys my project to google's servers whenever I push my code to github. It also automatically runs the tests I wrote to see if my code is broken (and if so, it doesn't push it).