\subsection{Discussion of Approaches and Ideas Considered and Rejected}
\paragraph{}
The first idea I came across was to make a website which combined the ideas of Sentral, NCSS challenge, and Openlearning (UNSW elearning website). However, I decided against it for a few reasons. Firstly, web development means you have to do a lot more sanity checking to make sure the user doesn't do anything too stupid, such as ensuring that fields are filled in correctly. With an application for an OS, that isn't nessecary because in a real time application such as the one I would be coding, I would only need to handle mouse input, and a few different keyboard commands to open different modes, all of which would always be valid.

\paragraph{}
The next idea I considered a few times was to rotate a camera around an object in order to create a 3d model of an object, but I never seriously considered it because it seemed like I would require a lot of equipment such as an arduino board, a stepper motor (which allows high precision movement), and a few other things.

\paragraph{}
After going to an Engineering open day at UNSW I decided I wanted to do a double degree in Computer Science and Mechatronics (Electrical and Mechanical Engineering), the perfect degree for robotics. After a while, I decided that I wanted to do a major work in robotics, so I could get a feel for whether or not I actually wanted to do Mechatronics. This lead me to come up with the idea of combining it with the previous idea to create a robot which utilises a SLAM (simultaneous localisation and mapping) system in order to navigate through its surroundings.

\paragraph{}
In the end I decided to use my original idea of a website due to the fact that it requires no resources, such as webcams, and is a lot easier to test, as I would be able to do test driven development with it (due to the fact that getting data off the robot was accuracy based, rather than simply wrong or right, it was impossible to do with a robot).